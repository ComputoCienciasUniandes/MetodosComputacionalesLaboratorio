\documentclass[11pt,letterpaper]{exam}
\usepackage{amsmath}
\usepackage[utf8]{inputenc}
\usepackage[spanish]{babel}
\usepackage{graphicx}
\usepackage{tabularx}
\usepackage[absolute]{textpos} % Para poner una imagen en posiciones arbitrarias
\usepackage{multirow}
\usepackage{float}
\usepackage{hyperref}
\usepackage{breakurl}
\decimalpoint

\begin{document}
\begin{center}

\includegraphics[width=16cm]{header.png}

\vspace{1.0cm}
{\Large Laboratorio de M\'etodos Computacionales - Ejercicio 3} \\
\textsc{Semana 8}\\
2017-I\\
\end{center}

\vspace{0.5cm}

\noindent
El ejercicio consiste de dos parte. La primera sobre aplicaciones de la transformada de Fourier y la segunda sobre soluciones num\'ericas de ecuaciones diferenciales ordinarias.
\section{Transformada de Fourier}
Los archivos del c\'odigo fuente debe subirse a Sicua plus en un \'unico archivo \verb'.zip' con el nombre del estudiante en el formato \verb"NombreApellido.zip" antes que termine la clase.
\noindent
\\El enlace a la imagen que vamos a utilizar es:
\url{http://www.scipy-lectures.org/_images/moonlanding.png}\\
\noindent
El archivo contiene una imagen del primer y \'unico alunizaje de la humanidad en 1969 gracias a la misi\'on Apolo 11.
El ejercicio consiste en modificar la imagen
\vspace{0.5cm}
\begin{questions}
\question[0.8] Escriba un script .sh que realice lo siguiente:
	\begin{itemize}
		\item (0.25 pts.) Cree un directorio llamado \verb'Alunizaje/.' y entre en el. 
		\item (0.25 pts.) Descargue el archivo \verb'moonlanding.png'.
		\item (0.25 pts.) Ejecute la rutina \verb'fourier_moonlanding.py'.
		\item (0.25 pts.) Elimine el archivo \verb'moonlanding.png'.
	\end{itemize}
\question[1.7] Escriba un script \verb'fourior_moonlanding.py' que realice lo siguiente:
	\begin{itemize}
		\item (0.5 pts.) Lea el archivo \verb'moonlanding.png' y guarde los datos necesarios.
		\item ()
	\end{itemize}
\end{questions}
\section{ODEs}
\begin{questions}
\question[0.8]
	\begin{itemize}
		\item (0.25 pts.)  
		\item (0.25 pts.) 
		\item (0.25 pts.) 
		\item (0.25 pts.) 
	\end{itemize}
\question[1.7]
\end{questions}

\end{document}
