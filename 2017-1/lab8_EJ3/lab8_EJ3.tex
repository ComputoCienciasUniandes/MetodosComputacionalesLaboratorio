\documentclass[11pt,letterpaper]{exam}
\usepackage{amsmath}
\usepackage[utf8]{inputenc}
\usepackage[spanish]{babel}
\usepackage{graphicx}
\usepackage{tabularx}
\usepackage[absolute]{textpos} % Para poner una imagen en posiciones arbitrarias
\usepackage{multirow}
\usepackage{float}
\usepackage{hyperref}
\usepackage{breakurl}
\decimalpoint

\begin{document}
\begin{center}

\includegraphics[width=16cm]{header.png}

\vspace{1.0cm}
{\Large Laboratorio de M\'etodos Computacionales - Ejercicio 2} \\
\textsc{Semana 7}\\
2017-I\\
\end{center}

%\begin{textblock*}{40mm}(10mm,20mm)
%  \includegraphics[width=3cm]{logoUniandes.png}
%\end{textblock*}

%\begin{textblock*}{40mm}(164mm,20mm)
%  \includegraphics[width=3cm]{logoUniandes.png}
%\end{textblock*}

\vspace{0.5cm}

\noindent
Los archivos del c\'odigo fuente debe subirse a Sicua plus en un \'unico archivo \verb'.zip' con el nombre del estudiante en el formato \verb"NombreApellido.zip" antes que termine la clase.

El enlace a los datos que vamos a utilizar es:

\url{http://openmv.net/file/room-temperature.csv}

El archivo contiene los datos de temperaturas de cuatro esquinas de un habitaci\'on en $144$ instantes de tiempo. Este ejercicio est\'a basado en el primer \'item de:

\url{https://learnche.org/pid/latent-variable-modelling/principal-component-analysis/pca-exercises}

\vspace{0.5cm}

\begin{questions}
 
\question[1.3]



\question[4.7]


\end{questions}

\end{document}
