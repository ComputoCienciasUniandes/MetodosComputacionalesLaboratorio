\documentclass[11pt,letterpaper]{exam}
\usepackage{amsmath}
\usepackage[utf8]{inputenc}
\usepackage[spanish]{babel}
\usepackage{graphicx}
\usepackage{tabularx}
\usepackage[absolute]{textpos} % Para poner una imagen en posiciones arbitrarias
\usepackage{multirow}
\usepackage{float}
\usepackage{hyperref}
\usepackage{breakurl}
\decimalpoint

\begin{document}
\begin{center}

\includegraphics[width=16cm]{header.png}

\vspace{1.0cm}
{\Large Laboratorio de M\'etodos Computacionales - Ejercicio 2} \\
\textsc{Semana 7}\\
2017-I\\
\end{center}

%\begin{textblock*}{40mm}(10mm,20mm)
%  \includegraphics[width=3cm]{logoUniandes.png}
%\end{textblock*}

%\begin{textblock*}{40mm}(164mm,20mm)
%  \includegraphics[width=3cm]{logoUniandes.png}
%\end{textblock*}

\vspace{0.5cm}

\noindent
Los archivos del c\'odigo fuente debe subirse a Sicua plus en un \'unico archivo \verb'.zip' con el nombre del estudiante en el formato \verb"NombreApellido.zip" antes que termine la clase.

El enlace a los datos que vamos a utilizar es:

\url{http://openmv.net/file/room-temperature.csv}

El archivo contiene los datos de temperaturas de cuatro esquinas de un habitaci\'on en $144$ instantes de tiempo. Este ejercicio est\'a basado en el primer \'item de:

\url{https://learnche.org/pid/latent-variable-modelling/principal-component-analysis/pca-exercises}

\vspace{0.5cm}

\begin{questions}
 
\question[1.3]

Escribir un script \verb'.sh' que realice lo siguiente:

\begin{itemize}
	\item (0.2 pts.) Descargue el archivo de datos \verb'room-temperature.csv'
	\item (0.2 pts.) Cree un directorio llamado \verb'Dir_Room/'.
	\item (0.2 pts.) Mueva a dicho directorio el archivo de datos y \textbf{copie} a \'el la rutina de Python.
	\item (0.2 pts.) Entrar a dicho directorio.
	\item (0.2 pts.) Corra la rutina de python \verb'pca_room.py'.
	\item (0.3 pts.) Elimine el directorio creado y los archivos que contiene.
\end{itemize}

\question[4.7]

El script \verb'pca_room.py' debe realizar lo siguiente:

\begin{itemize}
	\item (0.5 pts.) Leer el archivo \verb'room_temperature' y guardar los datos de las cuatro temperaturas. \textbf{No es necesario recuperar los datos de tiempo}.
	\item (1.0 pts.) Graficar temperatura vs. tiempo para cada una de las cuatro esquinas en gr\'aficas diferentes pero en la misma figura (como se ve en el enlace del ejercicio en \verb'learnche.org'). Guardar la figura en el archivo \verb'room.pdf'. Esta gr\'afica debe ser clara, con ejes debidamente rotulados. \textbf{No se preocupen por las unidades en el tiempo ni en la temperatura.}
	\item (0.5 pts.) Normalizar los datos de tal forma que las variables tengan media $0$ y varianza $1$. \textbf{Los puntos siguientes deben realizarse sobre los datos normalizados}.
	\item (0.3 pts.) Calcular e imprimir la matriz de covarianza para los datos luego de un mensaje que diga:

\begin{verbatim}
	La matriz de covarianza es:
\end{verbatim}

	\item (0.5 pts.) Obtener e imprimir en la consola las DOS componentes principales y sus correspondientes valores en orden descendente de valores. Se debe mostrar un mensaje como el siguiente:

\begin{verbatim}
	La primera componente principal es 'VECTOR1' con valor 'VALOR1'
	La segunda componente principal es 'VECTOR2' con valor 'VALOR2'
\end{verbatim}

donde \verb'VECTOR1', \verb'VECTOR2', \verb'VALOR1' y \verb'VALOR2' corresponden a los vectores y valores correspondientes a las componentes principales.

\item (0.5 pts.) Imprimir la contribuci\'on a la varianza de cada una de las dos primeras dos componentes principales en un mensaje como el siguiente:

\begin{verbatim}
	La primera componente principal explica el 'VAR1' % de la varianza.
	La segunda componente principal explica el 'VAR2' % de la varianza.
\end{verbatim}

donde \verb'VAR1', \verb'VAR2' es el porcentaje de la varianza que explican la primera y segunda componente principal, respectivamente.

\item (1.5 pts.) Graficar mediante \verb'scatter' los datos de las esquinas \verb'Front Right' vs. \verb'Front Left', y las esquinas \verb'Back Left' vs \verb'Front Left' incluyendo en cada una las rectas que representan las dos componentes principales. Estas gr\'aficas deben ser claras, con ejes debidamente rotulados. Guardar las gr\'afica en los archivos \verb'pca_fr_fl.pdf' y \verb'pca_bl_fl.pdf', respectivamente.
\end{itemize}

\end{questions}

\end{document}
