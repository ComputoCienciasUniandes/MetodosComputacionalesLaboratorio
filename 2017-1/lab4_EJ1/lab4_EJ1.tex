\documentclass[11pt,letterpaper]{exam}
\usepackage{amsmath}
\usepackage[utf8]{inputenc}
\usepackage[spanish]{babel}
\usepackage{graphicx}
\usepackage{tabularx}
\usepackage[absolute]{textpos} % Para poner una imagen en posiciones arbitrarias
\usepackage{multirow}
\usepackage{float}
\usepackage{hyperref}
\usepackage{breakurl}
\decimalpoint

\begin{document}
\begin{center}

\includegraphics[width=16cm]{header.png}

\vspace{1.0cm}
{\Large Laboratorio de M\'etodos Computacionales - Ejercicio 1} \\
\textsc{Semana 4}\\
2017-I\\
\end{center}

%\begin{textblock*}{40mm}(10mm,20mm)
%  \includegraphics[width=3cm]{logoUniandes.png}
%\end{textblock*}

%\begin{textblock*}{40mm}(164mm,20mm)
%  \includegraphics[width=3cm]{logoUniandes.png}
%\end{textblock*}

\vspace{0.5cm}

\noindent
Los archivos del c\'odigo fuente debe subirse a Sicua plus en un \'unico archivo \verb'.zip' con el nombre del estudiante en el formato \verb"NombreApellido.zip" antes que termine la clase.

El enlace al archivo que vamos a utilizar es:

\url{https://raw.githubusercontent.com/ComputoCienciasUniandes/MetodosComputacionalesLaboratorio/master/2017-1/lab4_EJ1/red3.txt}

El archivo contiene los datos de longitud de onda en nan\'ometros (columna 1) e intensidad (columna 2) para un diodo l\'aser.

\vspace{0.5cm}

\begin{questions}
 
\question[1.5]

Escribir un script \verb'.sh' que realice lo siguiente:

\begin{itemize}
	\item (0.1 pts.) Descargue el archivo de datos \verb'red3.txt'
	\item (1 pts.) Genere un archivo \verb'red3_filtrado.txt' que contenga todas las filas de \verb'red3.txt' donde la intensidad es mayor a $2000$.
	\item (0.2 pts.) Ejecute el script de Python \verb'plots_laser.py'.
	\item (0.2 pts.) Borre el archivo de datos \verb'red3.txt' y el archivo \verb'red3_filtrado.txt'.
\end{itemize}

\question[3.5]

El script \verb'plots_laser.py' debe realizar lo siguiente

\begin{itemize}
	\item (0.5 pts.) Leer el archivo \verb'red3_filtrado.txt' y guardar la longitud de onda y la intensidad en arreglos.
	\item (0.5 pts.) Graficar intensidad vs. longitud de onda y guardar la gr\'afica en el archivo \verb'red3.png'.
	\item (0.5 pts.) Imprimir la longitud de onda a la cual la amplitud es m\'axima. Mostrarlo en un mensaje de la siguiente manera:
		\begin{verbatim}
		La longitud de onda de maxima amplitud es n nm
		\end{verbatim}
	donde \verb'n' es el n\'umero que obtengan.
	\item (1 pts.) Realizar un fit a una funci\'on Gaussiana de la forma $y(x) = a\cdot e^{-(x-b)^2/c}$, se sugiere usar \verb'scipy.optimize.curve_fit'.
	\item (0.5 pts.) Graficar en una misma figura la intensidad vs. longitud de onda (gr\'afica anterior) y el resultado del ajuste. Deben verificar que la funci\'on obtenida efectivamente ajuste los datos. Guardar la gr\'afica en el archivo \verb'red3_fit.png'
	\item (0.5 pts.) Imprimir la longitud de onda a la cual la funci\'on ajustada es m\'axima. Mostrarlo en un mensaje de la siguiente manera:
		\begin{verbatim}
		La longitud de onda de maxima amplitud segun el ajuste es n nm
		\end{verbatim}
	donde \verb'n' es el n\'umero que obtengan.
\end{itemize}

\end{questions}

\end{document}
