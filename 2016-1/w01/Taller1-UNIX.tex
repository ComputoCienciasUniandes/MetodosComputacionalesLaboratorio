\documentclass[11pt]{article}
\usepackage{enumerate}
\usepackage[hmargin=3.0cm,vmargin=2.25cm]{geometry}

\begin{document}

\begin{center}
\textsc{\LARGE Taller 1 - UNIX}\\
\textsc{\large Enero 20 }
\end{center}


El script  que de soluci\'on de esta tarea deben subirse a
trav\'es de sicuaplus antes de las 2:30 pm del mi\'ercoles 20 de Enero como
un \'unico archivo con el nombre
\verb"NombreApellidos_taller1.sh", por ejemplo yo deber\'ia subir un
archivo llamado \verb"GermanChaparro_taller1.sh".\\
\\
El archivo \verb"data.tar.gz" contiene datos sobre ventas de algunos productos.\\

Escriba un script que utilice los comandos vistos y que haga las siguientes acciones.:
\begin{itemize}
\item (10 puntos) Cree una carpeta que se llame \verb+test2+ y copie el archivo \verb|data.tar.gz| dentro de esta nueva carpeta. 
\item (10 puntos) Descomprima \verb|data.tar.gz|, qu\'e archivos hay en \verb|data|?
\item (10 puntos) Cu\'antas lineas, palabras y bytes hay en cada archivo dentro de \verb|data|?
\item (10 puntos) Encuentre el archivo con mayor peso en bytes y as\'ignele el nombre de \verb|datalong.csv| y al de menor \verb|datashort.csv|
\item (10 puntos) Imprima las primeras 5 lineas de \verb|datalong.csv| y las ultimas 20 de \verb|datashort.csv|
\item (20 puntos) Encuentre la palabra \verb|Spartina| y reemplazela por la palabra \verb|Tremontina| y cree un nuevo archivo que se llame \verb|datashort_Tremontina.csv|

\item (30puntos) Seleccione los datos que esten en el estado de Rio de Janeiro con menos de 300 ventas a este nuevo archivo nombrelo \verb+datashort_TremontinaRio.csv+ .
 
\end{itemize}

\end{document}
