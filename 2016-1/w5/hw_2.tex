
%--------------------------------------------------------------------
%--------------------------------------------------------------------
% Formato para los talleres del curso de Métodos Computacionales
% Universidad de los Andes
% 2015-10
%--------------------------------------------------------------------
%--------------------------------------------------------------------

\documentclass[11pt,letterpaper]{exam}
\usepackage[utf8]{inputenc}
\usepackage[spanish]{babel}
\usepackage{graphicx}
\usepackage{tabularx}
\usepackage[absolute]{textpos} % Para poner una imagen completa en la portada
\usepackage{multirow}
\usepackage{float}
\usepackage{hyperref}
\decimalpoint
%\usepackage{pst-barcode}
%\usepackage{auto-pst-pdf}

\newcommand{\base}[1]{\underline{\hspace{#1}}}
\boxedpoints
\pointname{ pt}
%\extrawidth{0.75in}
%\extrafootheight{-0.5in}
\extraheadheight{-0.15in}
%\pagestyle{head}

%\noprintanswers
%\printanswers


\usepackage{upquote,textcomp}
\newcommand\upquote[1]{\textquotesingle#1\textquotesingle} % To fix straight quotes in verbatim

\begin{document}
\begin{center}
{\Large Laboratorio de Métodos Computacionales} \\
Taller 2 \\
Profesor: Germ\'an Chaparro\\
Fecha de Publicación: {\small \it Febrero 15 de 2016}\\
\end{center}

\begin{textblock*}{40mm}(10mm,20mm)
  \includegraphics[width=3cm]{logoUniandes.png}
\end{textblock*}

\begin{textblock*}{40mm}(161mm,20mm)
  \includegraphics[width=3cm]{logoUniandes.png}
\end{textblock*}

\vspace{0.5cm}

{\Large Instrucciones de Entrega}\\

\noindent
La solución a este taller debe subirse por sicuaplus antes de las 12:59PM
del martes 23 de febrero del 2016. 
\noindent
Debe entregarse un archivo llamado \verb"NombreApellido_hw2.ipynb".

\par

Todas los algoritmos deben ser implementados con funciones b\'asicas
de python, la función \verb"np.linalg.solve()", la librería \verb"numpy"
para arrays y la librería \verb"matplotlib"para realizar gráficas.

\begin{questions}

\question[20] {\bf{Temperatura en Munich}} en el archivo 
\url{http://xurl.es/munich.txt} se encuentra la temperatura promedio de 
la ciudad de Munich desde 1995 hasta 2013. 
Ajuste la función f(x) a los datos y grafique.

$$f(x) = a \cos \left( 2\pi t + b \right) + c $$

Es necesario eliminar los datos donde la temperatura alcance valores
+99 y -99.

\question[20]{\bf"Bondad del ajuste " $\chi^2 $} Para tener idea de qué tan bueno es el modelo, calcule la varianza del error cuadrático como la suma del cuadrado de la diferencia entre cada dato obsevado y el dato predicho por el ajuste.
$$ \chi^2 = \sum_{i=0}^{N} \frac{ \left(T_{\textrm{obs}} - T_\textrm{fit}
\right)^2}{N-1} $$

\question[20] {\bf{Calentamiento Global}}
 ¿Existe evidencia de calentamiento global? Si suponemos que la temperatura media ha aumentado en la última década, podemos añadir un término lineal.
Ajuste a los datos la función g(t) y grafique.

$$f(x) = k \cos \left( 2\pi t + w \right) + m\times t + p$$


\question[20]{\bf{Bondad del ajuste} $\chi_\textrm{cg}^2$}
Calcule de nuevo la varianza del error cuadrático en este nuevo modelo.
$$ \chi_\textrm{cg}^2 = \sum_{i=0}^{N} \frac{ \left(T_{\textrm{obs}} - T_\textrm{fit cg}\right)^2}{N-1} $$

\question[20]{\bf{Evidencia del calentamiento global}} ¿Cuál de los dos modelos tiene mejor ajuste?
\end{questions}


\end{document}
