%--------------------------------------------------------------------
%--------------------------------------------------------------------
% Formato para los talleres del curso de Métodos Computacionales
% Universidad de los Andes
%--------------------------------------------------------------------
%--------------------------------------------------------------------

\documentclass[11pt,letterpaper]{exam}
\usepackage[utf8]{inputenc}
%\usepackage[spanish]{babel}
\usepackage{graphicx}
\usepackage{tabularx}
\usepackage[absolute]{textpos} % Para poner una imagen en posiciones arbitrarias
\usepackage{multirow}
\usepackage{float}
\usepackage{hyperref}
\usepackage{url}
\usepackage{amsmath,amssymb}
\usepackage{bigints}
%\decimalpoint

\begin{document}
\begin{center}
{\Large M\'etodos Computacionales} \\
Tarea 4 - \textsc{Derivaci\'on num\'erica}\\
01-2018\\
\end{center}

\begin{textblock*}{40mm}(10mm,20mm)
  \includegraphics[width=3cm]{logoUniandes}
\end{textblock*}

\begin{textblock*}{40mm}(164mm,20mm)
  \includegraphics[width=3cm]{logoUniandes}
\end{textblock*}

\vspace{0.3cm}

\noindent
La soluci\'on a este taller debe subirse por SICUA antes de terminada la clase.
%\noindent
%Si la soluci\'on est\'a en SICUA
%antes de las 8:30AM del domingo 31 de Enero del 2016 se calificar\'a
%el taller sobre 125 puntos. 
\noindent
Los archivos c\'odigo fuente deben subirse en un \'unico archivo
\verb".zip" con el nombre \verb"NombreApellido_hw4.zip", por ejemplo
yo deber\'ia subir el zip \verb"JesusPrada_hw4.zip" (10 puntos). Recuerden que es un trabajo individual y debe ser realizado en un script de python (.py).

\vspace{0.3cm}

En el archivo \verb"planets.zip" encontrar\'an las coordenadas de los planetas del sistema solar desde el 2010. Cada archivo consta de 5 columnas con los valores de A\~no ,D\'ia,Radio(UA) ,Latitud(deg) y Longitud(deg), respectivamente. El objetivo de este taller es que a partir de los datos de cada planeta, y usando derivaci\'on num\'erica, puedan estimar la masa del sol.\\

Pero antes, un poco de teor\'ia. La tercera ley de Kepler nace considerar el equilibrio entre el momento angular de un planeta y su respectiva fuerza centr\'ipeta por el sol. Esto nos lleva a la ecuaci\'on:

\begin{equation}
mr\omega^2 = \frac{GMm}{r^2},
\end{equation}

donde $r$ ha de tomarse como el semi-eje mayor de la elipse que describe el planeta.\\

Si queremos la masa del sol $M$, entonces tenemos la siguiente ecuaci\'on:

\begin{equation}
\frac{r^3\omega^2}{G} = M.
\end{equation}

Ahora, para obtener $\omega$, debemos usar los datos espec\'ificos de cada planeta teniendo en cuenta que:

\begin{equation}
\omega_\theta = cos(\phi)\dot{\theta},
\omega_\phi = \dot{\phi},
\end{equation}

donde $\theta$ es la latitud y  $\phi$ es la longitud.

\begin{questions}

\question[20] {\bf{Cargar los datos}}

Cargue los datos de cada planeta y obtenga, para cada uno, radio, latitud y longitud. Tengan cuidado con la distici\'on entre grados y radianes.

\question[50] {\bf{Derivaci\'on num\'erica}}

Cree una funci\'on que, dadas la latitud y longitud, calcule la velocidad angular usando la aproximaci\'on de derivada central. No se olvide que el intervalo de los datos es de un d\'ia.\\

Grafique, para cada planeta, la velocidad angular en funci\'on del tiempo.\\

\question[20] {\bf{La masa del sol}}

Omitiendo los valores correspondientes a saltos de \'angulo de 360 a 0 grados, calcule el promedio de la velocidad angular. Luego, usando ese promedio, y el semi-eje mayor, para cada planeta calcule la masa del sol y calcule el error obtenido en cada caso.

\question[10]{\bf{Comentarios}}

Comenten su c\'odigo.

\question[10]{Bono}

Comenten qu\'e errores v\'alidos pueden influir en los c\'alculos.

\end{questions}
\end{document}
