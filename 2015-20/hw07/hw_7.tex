%--------------------------------------------------------------------
%--------------------------------------------------------------------
% Formato para los talleres del curso de Métodos Computacionales
% Universidad de los Andes
% 2015-10
%--------------------------------------------------------------------
%--------------------------------------------------------------------

\documentclass[11pt,letterpaper]{exam}
\usepackage[utf8]{inputenc}
\usepackage[spanish]{babel}
\usepackage{graphicx}
\usepackage{tabularx}
\usepackage[absolute]{textpos} % Para poner una imagen completa en la portada
\usepackage{multirow}
\usepackage{float}
\usepackage{hyperref}
\usepackage{amsfonts}
\decimalpoint
%\usepackage{pst-barcode}
%\usepackage{auto-pst-pdf}

\newcommand{\base}[1]{\underline{\hspace{#1}}}
\boxedpoints
\pointname{ pt}
%\extrawidth{0.75in}
%\extrafootheight{-0.5in}
\extraheadheight{-0.15in}
%\pagestyle{head}

%\noprintanswers
%\printanswers


\usepackage{upquote,textcomp}
\newcommand\upquote[1]{\textquotesingle#1\textquotesingle} % To fix straight quotes in verbatim

\begin{document}
\begin{center}
{\Large Laboratorio de Métodos Computacionales} \\
Taller 7 \\
Profesor: Felipe G\'omez\\
Fecha de Publicación: {\small \it Noviembre 12 de 2015}\\
\end{center}

\begin{textblock*}{40mm}(10mm,20mm)
  \includegraphics[width=3cm]{logoUniandes.png}
\end{textblock*}

\begin{textblock*}{40mm}(161mm,20mm)
  \includegraphics[width=3cm]{logoUniandes.png}
\end{textblock*}

\vspace{0.5cm}

{\Large Instrucciones de Entrega}\\

\noindent
La solución a este taller debe subirse por SICUA antes de las 12:59PM
del s\'abado 14 de noviembre de 2015. Debe entregarse un archivo llamado
\verb"NombreApellido_hw7.ipynb". Este puede iniciar con \verb"%pylab inline"

\begin{questions}

\question[100] {\bf{Ajuste de modelo parabólico con MCMC}} 

En el repositorio de github se encuentra el repositorio visto en clase

\href{https://github.com/ComputoCienciasUniandes/MetodosComputacionales/blob/master/notes/14.MonteCarloMethods/bayes_MCMC.ipynb}{MetodosComputacionales/notes/14.MonteCarloMethods/bayes\_MCMC.ipynb}

\noindent junto con el archivo \verb"movimiento.dat". Realice un ajuste de los datos a un modelo parabólico de la forma 

$$ y_{fit}= A + B x + C x^2$$

utilizando el algoritmo de Cadena de Markov Monte-Carlo (MCMC) teniendo en cuenta:
\begin{itemize}
 \item La funci\'on de error est\'a definida como $\chi^2 = \sum _i ^n \left( Y_{obs} - Y_{fit} \right)^2$
 \item La funci\'on de verosimilitud est\'a escrita como $\mathcal{L} = \exp(-\chi^2)$
 \item El criterio de selecci\'on est\'a dado por
       $\alpha=\mathcal{L}_{new} / \mathcal{L}_{old} \geq 1.0$
 \item Si $\chi^2$ es grande, entonces $\mathcal{L}$ ser\'a muy peque\~no 
       y la m\'aquina lo puede hacer igual a cero. El criterio de 
       selecci\'on encontrar\'ia una divisi\'on del tipo $0/0$
 \item Se puede trabajar con un nuevo criterio de selecci\'on 
       que sea el logaritmo de $\alpha$ para aceptar de inmediato
       los nuevos par\'ametros.
       $$ \gamma = -\chi^2_{old}+\chi^2_{new} \geq 0  $$
 \item Si $\gamma<0$, generar el n\'umero aleatorio $\beta$ y verificar la condici\'on
       $$ \beta \leq \exp(\gamma) = \alpha $$
       Si se cumple aceptar de inmediato los nuevos par\'ametros, si no, rechazarlos.
\end{itemize}

Realizar las gr\'aficas de par\'ametros vs. $\chi^2$, histogramas de par\'ametros
y la gr\'afica del ajuste del modelo a los datos observacionales con los
mejores par\'ametros.

\end{questions}


\end{document}
